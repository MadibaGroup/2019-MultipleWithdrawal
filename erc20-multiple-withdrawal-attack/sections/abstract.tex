% !TEX root = ../main.tex

\begin{abstract}

Custom tokens are an integral part of Ethereum decentralized applications (DApps). The ERC20 standard defines a token interface that is interoperable with any ERC20-compliant application, which might be another decentralized application or a web application. A standard-level issue, known as the multiple withdrawal attack, was raised on Github and has been open since October 2017. It exploits the way ERC20 describes a method called approve, which is used by a token holder to approve other users or DApps to withdrawal a capped number of tokens. If the token holder makes adjustments to the amount of tokens approved, a malicious entity can front-run this new approval, withdraw the original allotment of tokens, and then wait for the new approval to be confirmed which provides a new allotment of tokens available for withdrawal. In this paper, we (1) examine ten suggested solutions in accordance with specifications of the standard and being backward compatible with already deployed smart contracts; and (2) propose a new solution that mitigates the attack.

\end{abstract}


\begin{IEEEkeywords}

Cryptocurrency; Security; ERC20; Token; Smart Contract; Front-Running; Ethereum; Blockchain; 

\end{IEEEkeywords}