% !TEX root = ../main.tex

\begin{abstract}

Custom tokens are an integral part of Ethereum decentralized applications (DApps). The ERC20 standard defines a token interface that allows any Ethereum address to have a balance of a secondary token, and is interoperable with any ERC20-compliant application. An issue, known as the \textit{multiple withdrawal attack}, was raised on GitHub and has been open since October 2017. It exploits ERC20 standard API describing a method called \texttt{approve()}, which is used by a token holder to approve other users or DApps to withdrawal a capped number of tokens. If the token holder makes adjustments to the amount of tokens approved, a malicious entity can front-run this new approval, withdraw the original allotment of tokens, and then wait for the new approval to be confirmed which provides a new allotment of tokens available for withdrawal. In this paper, we (1) examine \textit{ten} suggested solutions in accordance with specifications of the standard and being backward compatible with already deployed smart contracts; and (2) propose two new solutions that mitigate the attack.

\end{abstract}


\begin{IEEEkeywords}

Cryptocurrency; Security; ERC20; Token; Smart Contract; Front-Running; Ethereum; Blockchain; 

\end{IEEEkeywords}