% !TEX root = ../main.tex

\begin{abstract}

ERC20 standard defines set of interfaces for standardizing interaction with tokens on the Ethereum blockchain. Tokens in Ethereum ecosystem facilitate creation of digital assets by introducing standard functions that can be reused by ERC20-compliant applications. Being as a subset of smart contracts, makes them vulnerable to security flaws. Particularly, two functions in ERC20 standard allow token transfer on behalf of the owner. Using these two functions in case of front-running could lead to "Multiple Withdrawal Attack" that allows a spender to transfer more tokens than the owner ever wanted. This standard-level issue was initially raised on Github and may even impact security of already deployed tokens. Openness of the issue since October 2017, motivated us to (1) examine ten suggested solutions in accordance with specifications of the standard and being backward compatible with already deployed smart contracts; (2) propose a new solution that mitigates the attack sustainably.

\end{abstract}


\begin{IEEEkeywords}

Cryptocurrency; Security; ERC20; Token; Smart Contract; Ethereum; Blockchain; 

\end{IEEEkeywords}