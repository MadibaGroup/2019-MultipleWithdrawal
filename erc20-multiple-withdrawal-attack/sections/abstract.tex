% !TEX root = ../main.tex

\begin{abstract}

Custom tokens are an integral component of decentralized applications (dapps) deployed on Ethereum and other blockchain platforms. For Ethereum, the ERC20 standard is a widely used token interface and is interoperable with many existing dapps, user interface platforms, and popular web applications (\eg exchange services). An ERC20 security issue, known as the \textit{multiple withdrawal attack}, was raised on GitHub and has been open since October 2017. The issue concerns ERC20's defined method  \texttt{approve()} which was envisioned as a way for token holders to give permission for other users and dapps to withdraw a capped number of tokens. The security issue arises when a token holder wants to adjust the amount of approved tokens from $N$ to $M$. If malicious, a user or dapp who is approved for $N$ tokens can front-run the adjustment transaction to first withdraw $N$ tokens, then allow the approval to be confirmed, and withdraw an additional $M$ tokens. In this paper, we evaluate 10 proposed mitigations for this issues and find that no solution is fully satisfactory. We then propose 2 new solutions that mitigate the attack, one of which fully fulfills constraints of the standard, and the second one shows a general limitation in addressing this issue from ERC20's approve method.

\end{abstract}


\begin{IEEEkeywords}

Ethereum; ERC20 tokens; Blockchain;

\end{IEEEkeywords}