\begin{abstract}

ERC20 standard defines set of interfaces for standardizing interaction with tokens on the Ethereum blockchain. Tokens facilitate creation of digital assets by introducing standard functionalities that can be reused by ERC20-compliant applications. Being subset of smart contracts, makes tokens vulnerable to security flaws, particularly, two functions in the standard that allow token transfer on behalf of the owner. In case of front-running, these two functions could be used in "Multiple Withdrawal Attack" that allows a spender to transfer more tokens than the owner ever wanted. This standard-level issue was initially raised on Github and may impact security of already deployed smart contracts. Its openness since October 2017 motivated us to (1) examine ten suggested solutions in terms of adhering to the specifications of standard and being backward compatible; (2) proposing a new solution that mitigates sustainably the attack.

\end{abstract}


\begin{IEEEkeywords}

Cryptocurrency; Security; ERC20; Token; Smart Contract; Ethereum; Blockchain; 

\end{IEEEkeywords}