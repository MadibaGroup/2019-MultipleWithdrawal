% !TEX root = ../main.tex

\linespread{1.25}
\section{Conclusion}
Based on ERC20 specifications, token owners should be aware of their approval consequences. If they approve someone to transfer N tokens, the spender can transfer exactly N tokens, even if they change allowance to zero afterward. This is considered a legitimate transaction and responsibility of approver before allowing the spender for transferring any tokens. An attack can happen when changing allowance from N to M, that allows spender to transfer N+M tokens and effect multiple withdrawal attack. This attack is possible in case of front-running by approved side. As we examined possible solutions, all approaches violate ERC20 specifications or have not addressed the attack completely. Proposal 1 uses CAS pattern for checking and setting new allowance atomically. In proposal 2, \textit{transferFrom} function is secured instead of \textit{approve} method. We implemented an ERC20 token for each proposal that solve this security issue while keeping backward compatibly with already deployed smart contracts or wallets. Although these implementations consume more Gas than standard ERC20 implementations, they are secure and could be considered for secure ERC20 token deployment.
