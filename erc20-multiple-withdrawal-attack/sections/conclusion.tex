% !TEX root = ../main.tex

\section{Conclusion}
Based on ERC20 specifications, token owners should be aware of their approval consequences. If they approve someone to transfer N tokens, the spender can transfer exactly N tokens, even if they change allowance to zero afterwards. This is considered a legitimate transaction and responsibility of the approver before allowing the spender for transferring any tokens. Multiple withdrawal attack can occur when allowance changes from N to M, that allows spender to transfer N+M tokens in total. This attack is possible in case of front-running by approved side. As we examined possible solutions, all approaches violate ERC20 specifications or have not addressed the attack effectively. In this paper we introduced two proposals for securing vulnerable methods---\textit{approve} and \textit{transferFrom}. Proposal 1 incorporates CAS pattern for comparing and setting new allowance atomically. It adjusts new allowance based on transferred tokens which is violating one of ERC20 constraint that says setting new allowance instead of adjusting it. As discussed securing \textit{approve} method is not feasible while adhering ERC20 specifications. Therefore, we secured \textit{transferFrom} function instead of \textit{approve} method in the second proposal. Each proposal has been implemented on Rinkeby network and tested in terms of backward compatibly with already deployed smart contracts and conforming with the standard. Although new proposals consume more gas compared to standard ERC20 implementations, they are secure and could be considered for future secure ERC20 token deployments.
