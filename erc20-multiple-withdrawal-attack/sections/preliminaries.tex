% !TEX root = ../main.tex

\section{Preliminaries}

\subsection{How \textit{multiple withdrawal attack} works}

According to the ERC20 API definition:

\begin{compactlist}
\item The \texttt{approve} function\footnote{Syntax of approve method: \texttt{approve(address \texttt{\_spender}, uint256 \texttt{\_tokens})}} allows \texttt{\_spender} to withdraw up to the \texttt{\_value} amount of tokens from token pool of the approver. If this function is called again, it overwrites the current allowance with the new \texttt{\_value}.
\item The \texttt{transferFrom} function\footnote{Syntax of transferFrom method: \texttt{transferFrom(address \texttt{\_from}, address \texttt{\_to}, uint256 \texttt{\_tokens})}} allows the spender (\eg account, wallet or other smart contract) to transfer \texttt{\_value} amount of tokens on behalf of address \texttt{\_from} to address \texttt{\_to}. It updates balance of parties in the transaction accordingly. 
\end{compactlist}

An adversary can exploit the gap between execution of \texttt{approve} and \texttt{transferFrom} functions since the \texttt{approve} method overrides current allowance regardless of whether the spender already transferred any tokens or not. Futher, transferred tokens are not trackable and only \texttt{Transfer} events\footnote{Syntax of Transfer event: \texttt{Transfer(address indexed \texttt{\_from}, address indexed \texttt{\_to}, uint256 \texttt{\_value})}} are logged, which is not sufficient in case of transferring tokens to a third party (\texttt{Transfer} event is not related to Bob). To make it more clear, the following attack scenario can be considered:

% Jeremy: Not sufficient for what? Above

\begin{compactlistn}
	\item Alice allows Bob to transfer N tokens on her behalf by calling \texttt{approve(\_Bob, N)}.
	\item Later, Alice decides to change Bob's approval from N to M by executing \texttt{approve(\_Bob, M)}.
	\item Bob notices Alice's second transaction after it is broadcast to the Ethereum network but before it is added to a block. 
	\item Bob using the asymmetric insertion method~\cite{eskandari2019sok} front-runs the original transaction to run  \texttt{transferFrom(\_Alice, \_Bob, N)} first. If a miner adds this transaction before Alice's, it will transfer N Alice's tokens to Bob.
	\item Alice's transaction will be executed after Bob's.
	\item Bob can call \texttt{transferFrom} method again and transfers M additional tokens by executing \texttt{transferFrom(\_Alice, \_Bob, M)}.
\end{compactlistn}

In summary, in attempting to change Bob's allowance from N to M, Alice makes it possible for Bob to transfer N+M of her tokens. We operate on the assumption that a secure implementation would prevent Bob from withdrawing Alice's tokens multiple times when the allowance changes from N to M. If Bob could withdraw N tokens after Alice initial approval, this would be considered as legitimate transfer, since Alice has already approved it. In other words, it would be the responsibility of Alice to make sure tokens have not been transferred before modifying her approval for Bob.

\subsection{General prevention techniques}

Solutions can be categorized into two categories:
\begin{compactlist}
	\item \textbf{Prevention by token owner}: Considering owner responsibility to mitigate the attack in UI (\eg Web3.js) instead of contract (\ie Solidity/Vyper).
	\item \textbf{Prevention by token author}: Smart contract author have to mitigate the attack by securing either \texttt{approve} or \texttt{transferFrom} method.
\end{compactlist}

\subsubsection*{Prevention by token owner} This approach is recommended by ERC20 authors \cite{Ref08} and advises owner to change spender allowance from N to zero before zero to M (instead of directly from N to M). This proposed method, will not prevent the front-running attack but it is a work around to the specific multiple withdrawal attack scenario. In the smart contract level, change of allowance from N to zero by the owner is indistinguishable from transferring the whole balance by the attacker (\texttt{transferFrom}).
\newline Although it would be possible to track transferred token through \texttt{Transfer} events, but it would miss some information in case the transfer happens to a third-party. For example, if Alice allows Bob and then Bob transfers tokens to Carol, \texttt{Transfer} event creates a log showing Carol moved tokens from Alice. As discussed later in~\ref{sec:MiniMeToken}, this approach can not prevent the attack since it is not distinguishable which transaction (\ie owner or spender transaction) has set the allowance to zero before non-zero values.

%SHAYAN: maybe we should separate (or mention) off-chain detections (e.g events) and on-chain detection. in order to prevent such attacks I believe on-chain solutions are required. however I have to go through all the proposals to have a more complete opinion about this.
	
\subsubsection*{Prevention by \texttt{approve} method} By using compare and set (CAS) pattern \cite{Ref06}, \texttt{approve} method can change spender allowance from N to M atomically (\ie comparing new allowance with transferred token and set it accordingly). Comparison part of CAS requires knowledge of previously transferred tokens that will reveal any token transfer in case of front-running. Although this is promising approach,  setting new allowance in \texttt{approve} method must satisfy ERC20 constraint that dictates "If this function is called again it overwrites the current allowance with \texttt{\_value}" \cite{Ref08}. In other words, any adjustments in allowance is prohibited, however this is a prerequisite of securing \texttt{approve} method. 
%SHAYAN: REZA I'm not sure what this example is.
For example, considering front-running by Bob when Alice changes his allowance form 100 to 110, the \texttt{approve} method can reveal 100 transferred tokens by Bob and has to set Bob allowance to 10 (110-100=10). However, based on ERC20 constraints, it must not adjust new allowance and has to set it literally to 110. Consequently, this allows Bob to transfer 100+110=210 tokens in total. We implemented this approach in proposal 1 and concluded that securing \texttt{approve} method cannot prevent the attack while adhering constraints of the ERC20 standard.
	
\subsubsection*{Prevention by \texttt{transferFrom} method} Based on ERC20 constraints, "approve functions allows \texttt{\_spender} to withdraw from your account multiple times, up to the \texttt{\_value}". Therefore, spender must not be able to transfer more than authorized tokens. That being said, \texttt{transferFrom} method can prevent transferring of M new tokens in case of already transferred N tokens. By comparing transferred tokens in \texttt{transferFrom} method, spender will be restricted to move solely the remaining tokens of his allowance. In case of trying to transfer more tokens than allowed, the transaction fails. For example, Alice's new transaction for increasing Bob allowance from 100 to 110, sets Bob allowance to 110 (\texttt{approve} method is still insecure and sets allowance regardless of transferred tokens). However, \texttt{transferFrom} method prevent the attack by not allowing Bob to transfer more than 10 tokens if he had already transferred 100 tokens. We implemented this approach in proposal 2 and it mitigates the attack effectively.

\subsection{Properties of acceptable solutions}
An important criterion for a solution is to adhere the specifications of ERC20 standard. Conforming with the standard ensures that new tokens are backward-compatible with already deployed ERC20 smart contracts and web applications. We summarize defined constraints from ERC20 specifications \cite{Ref08} that must be satisfied by any sustainable solution:


%SHAYAN: REZA isn't the second item the same as the first item? 
\begin{enumerate}
	\item Calling \texttt{approve} function has to overwrite current allowance with new allowance.
	\item \texttt{approve} method does not adjust allowance, it sets new value of allowance.
	\item Transferring 0 values by \texttt{transferFrom} method MUST be treated as normal transfers and fire the \texttt{Transfer} event as non-zero transactions.
	\item Introducing new methods violate ERC20 API and MUST be avoided for having compatible token with already deployed smart contracts. In other words, introducing new secure methods that are not defined in the initial ERC20 API, make them unusable with already deployed smart contracts. Therefore, smart contracts cannot use them to prevent the attack.
	\item Spender will be allowed to withdraw from approver account multiple times, up to the allowed amount.
	\item Transferring initial allowed tokens is considered as legitimate transfer. It could happen right after approval or before changing allowance.
	\item Race condition MUST not happen in any case to prevent multiple withdrawal from the account.
\end{enumerate}