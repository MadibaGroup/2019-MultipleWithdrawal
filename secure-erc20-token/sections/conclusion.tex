% !TEX root = ../main.tex

\section{Conclusion}

This paper focuses on the ERC20 tokens; Ethereum's most popular token standard. The development of smart contracts has proven to be error-prone in practice, and as a result, contracts deployed on public platforms are often riddled with security vulnerabilities. Exploited by the attackers, these vulnerabilities can often lead to major security incidents which introduce great cost due to the immutability characteristics of the blockchain technology. In this paper, we carefully select three significant vulnerabilities based on their attack vector and broader impact on the Ethereum blockchain. We thoroughly discuss the technical details of the vulnerabilities, the circumstances of the incidents together with their impacts, mitigations, and the broader lessons we learn from these incidents. Eventually, we propose a secure version of ERC20 token which is not vulnerable to any of the attack discussed.



