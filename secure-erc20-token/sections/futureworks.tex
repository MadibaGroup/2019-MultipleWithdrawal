% !TEX root = ../main.tex

\section{Future Works}
As future works, we are interested to see how many ERC20 tokens are impacted by one of these attacks. It requires (i) creating a full Ethereum node that have local copy of the blockchain for faster look up (ii) developing a script to scan the blockchain for ERC20 tokens (iii) using a complementary script to convert byte codes to \textit{Solidity} human readable codes (iv) searching for vulnerable codes and discovering impacted tokens. After finding vulnerable ERC20 tokens, the question would be how to remediate them. Remediation could be in form of migration since after deploying an ERC20 token on the blockchain, it becomes immutable. So, if the token contract being vulnerable to one of these security threads, developers needs to create a new ERC20 token and migrate users to the new one. The new token will have different address and revised code to fix identified vulnerabilities\footnote{There would be one exception in case of using upgradable ERC20 tokens. Developers can point the base contract to the new address and make the migration transparent for users. Although having upgradable ERC20 token would have some trade-offs that needs to be considered (\textit{i.e.,} risks of the data separation pattern, risks of delegatecall-based proxies, \textit{etc.}).}. Nonetheless, migration of ERC20 code to a secure version would not be convenient due to several technical considerations (\textit{i.e.,} transaction cost, data structure, data recovery, number of token holders, \textit{etc.}). This also includes updating any trading platform listing ERC20 tokens. Thus, it would be extremely challenging to estimate associated costs with this migration. Formalization of it could be another topic to discover in further research.
