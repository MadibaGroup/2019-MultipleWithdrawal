% !TEX root = ../main.tex

\section{Audit Tools}\label{section:tools}

% !TEX root = ../main.tex

\newcommand{\BP}{\texttt{BP}}
\newcommand{\fp}{$\oplus$}
\newcommand{\fail}{$\times$}
\newcommand{\noSWC}{$\bigcirc$}
\newcommand{\info}{$!$}
\newcommand{\pass}{$\checkmark$}
\newcommand{\na}{}
\newcommand{\tx}[1]{\fontfamily{cmss}\selectfont{\textbf{#1}}}
\newcommand{\ccl}{\cellcolor[HTML]{EFEFEF}}
\newcommand{\rcl}{\rowcolor[HTML]{EFEFEF}}
\newcolumntype{P}[1]{>{\centering\arraybackslash}p{#1}}

\begin{table*}
\centering
\begin{adjustbox}{max height=10cm}
\begin{tabular}{|P{4mm}|P{8mm}|m{105mm}|P{5mm}|P{5mm}|P{5mm}|P{5mm}|P{5mm}|P{5mm}|P{5mm}|}

\multicolumn{3}{c}{\.} &
\headrow{\tx{EY Token Review}} &
\headrow{\tx{Smart Check}} &
\headrow{\tx{Securify}} &
\headrow{\tx{MythX (Mythril)}} &
\headrow{\tx{Contract Guard}} &
\headrow{\tx{Slither}} &
\headrow{\tx{Odin}} \\ \hline

\rcl\ccl & \ccl & \tx{Vulnerability or best practice} & \multicolumn{7}{c|}{\ccl} \\ \cline{3-3}
\rcl\multirow{-2}{*}{\ccl\tx{ID}} & \multirow{-2}{*}{\ccl\tx{SWC}} & Mitigation or recommendation & \multicolumn{7}{c|}{\multirow{-2}{*}{\ccl\tx{Security tools}}} \\ \hline

\multirow{2}{*}{1} & \multirow{2}{*}{100} & \tx{Function default visibility} & \multirow{2}{*}{\na} & \multirow{2}{*}{\pass} & \multirow{2}{*}{\na} & \multirow{2}{*}{\pass} & \multirow{2}{*}{\pass} & \multirow{2}{*}{\na} & \multirow{2}{*}{\pass} \\ \cline{3-3} & & Specifying function visibility, external, public, internal or private & & & & & & & \\ \hline
\multirow{2}{*}{2} & \multirow{2}{*}{101} & \tx{Integer Overflow and Underflow} & \multirow{2}{*}{\fp} & \multirow{2}{*}{\info} & \multirow{2}{*}{\na} & \multirow{2}{*}{\pass} & \multirow{2}{*}{\pass} & \multirow{2}{*}{\na} & \multirow{2}{*}{\pass} \\ \cline{3-3}
& & Utilizing the SafeMath library to mitigate over/under value assignments & & & & & & & \\ \hline
\multirow{2}{*}{3} & \multirow{2}{*}{102} & \tx{Outdated Compiler Version} & \multirow{2}{*}{\pass} & \multirow{2}{*}{\pass} & \multirow{2}{*}{\pass} & \multirow{2}{*}{\pass} & \multirow{2}{*}{\pass} & \multirow{2}{*}{\pass} & \multirow{2}{*}{\fail} \\ \cline{3-3}
& & Using proper Solidity version to protect against compiler attacks & & & & & & & \\ \hline
\multirow{2}{*}{4} & \multirow{2}{*}{103} & \tx{Floating Pragma} & \multirow{2}{*}{\na} & \multirow{2}{*}{\pass} & \multirow{2}{*}{\pass} & \multirow{2}{*}{\pass} & \multirow{2}{*}{\na} & \multirow{2}{*}{\pass} & \multirow{2}{*}{\pass} \\ \cline{3-3}
& & Specifying function visibility & & & & & & & \\ \hline
\multirow{2}{*}{5} & \multirow{2}{*}{104} & \tx{Unchecked Call Return Value} & \multirow{2}{*}{\fp} & \multirow{2}{*}{\na} & \multirow{2}{*}{\pass} & \multirow{2}{*}{\pass} & \multirow{2}{*}{\pass} & \multirow{2}{*}{\pass} & \multirow{2}{*}{\pass} \\ \cline{3-3}
& & Checking call() return value & & & & & & & \\ \hline
\multirow{2}{*}{6} & \multirow{2}{*}{105} & \tx{Unprotected Ether Withdrawal} & \multirow{2}{*}{\na} & \multirow{2}{*}{\info} & \multirow{2}{*}{\na} & \multirow{2}{*}{\pass} & \multirow{2}{*}{\na} & \multirow{2}{*}{\pass} & \multirow{2}{*}{\pass} \\ \cline{3-3}
& & Authorizing trusted parties & & & & & & & \\ \hline
\multirow{2}{*}{7} & \multirow{2}{*}{106} & \tx{Unprotected SELFDESTRUCT Instruction} & \multirow{2}{*}{\na} & \multirow{2}{*}{\na} & \multirow{2}{*}{\pass} & \multirow{2}{*}{\pass} & \multirow{2}{*}{\na} & \multirow{2}{*}{\pass} & \multirow{2}{*}{\pass} \\ \cline{3-3}
& & Approving by multiple parties & & & & & & & \\ \hline
\multirow{2}{*}{8} & \multirow{2}{*}{107} & \tx{Re-entrancy} & \multirow{2}{*}{\na} & \multirow{2}{*}{\pass} & \multirow{2}{*}{\fp} & \multirow{2}{*}{\fp} & \multirow{2}{*}{\fp} & \multirow{2}{*}{\pass} & \multirow{2}{*}{\pass} \\ \cline{3-3}
& & Using CEI or Mutex & & & & & & & \\ \hline
\multirow{2}{*}{9} & \multirow{2}{*}{108} & \tx{State variable default visibility} & \multirow{2}{*}{\pass} & \multirow{2}{*}{\info} & \multirow{2}{*}{\pass} & \multirow{2}{*}{\pass} & \multirow{2}{*}{\pass} & \multirow{2}{*}{\na} & \multirow{2}{*}{\pass} \\ \cline{3-3}
& & Specifying variable visibility & & & & & & & \\ \hline
\multirow{2}{*}{10} & \multirow{2}{*}{109} & \tx{Uninitialized Storage Pointer} & \multirow{2}{*}{\na} & \multirow{2}{*}{\na} & \multirow{2}{*}{\pass} & \multirow{2}{*}{\pass} & \multirow{2}{*}{\pass} & \multirow{2}{*}{\pass} & \multirow{2}{*}{\pass} \\ \cline{3-3}
& & Initializing upon declaration & & & & & & & \\ \hline
\multirow{2}{*}{11} & \multirow{2}{*}{110} & \tx{Assert Violation} & \multirow{2}{*}{\na} & \multirow{2}{*}{\pass} & \multirow{2}{*}{\na} & \multirow{2}{*}{\pass} & \multirow{2}{*}{\na} & \multirow{2}{*}{\na} & \multirow{2}{*}{\pass} \\ \cline{3-3}
& & Using require() statement & & & & & & & \\ \hline
\multirow{2}{*}{12} & \multirow{2}{*}{111} & \tx{Use of Deprecated Solidity Functions} & \multirow{2}{*}{\na} & \multirow{2}{*}{\pass} & \multirow{2}{*}{\na} & \multirow{2}{*}{\pass} & \multirow{2}{*}{\pass} & \multirow{2}{*}{\pass} & \multirow{2}{*}{\pass} \\ \cline{3-3}
& & Using new alternatives & & & & & & & \\ \hline
\multirow{2}{*}{13} & \multirow{2}{*}{112} & \tx{Delegatecall to untrusted callee} & \multirow{2}{*}{\na} & \multirow{2}{*}{\fp} & \multirow{2}{*}{\fp} & \multirow{2}{*}{\pass} & \multirow{2}{*}{\pass} & \multirow{2}{*}{\pass} & \multirow{2}{*}{\pass} \\ \cline{3-3}
& & Using for trusted contracts & & & & & & & \\ \hline
\multirow{2}{*}{14} & \multirow{2}{*}{113} & \tx{DoS with Failed Call} & \multirow{2}{*}{\na} & \multirow{2}{*}{\pass} & \multirow{2}{*}{\na} & \multirow{2}{*}{\pass} & \multirow{2}{*}{\pass} & \multirow{2}{*}{\na} & \multirow{2}{*}{\pass} \\ \cline{3-3}
& & Avoid multiple external calls & & & & & & & \\ \hline
\multirow{2}{*}{15} & \multirow{2}{*}{114} & \tx{Transaction Order Dependence} & \multirow{2}{*}{\fp} & \multirow{2}{*}{\na} & \multirow{2}{*}{\pass} & \multirow{2}{*}{\pass} & \multirow{2}{*}{\na} & \multirow{2}{*}{\na} & \multirow{2}{*}{\pass} \\ \cline{3-3}
& & Preventing race conditions & & & & & & & \\ \hline
\multirow{2}{*}{16} & \multirow{2}{*}{115} & \tx{Authorization through tx.origin} & \multirow{2}{*}{\pass} & \multirow{2}{*}{\pass} & \multirow{2}{*}{\pass} & \multirow{2}{*}{\pass} & \multirow{2}{*}{\pass} & \multirow{2}{*}{\pass} & \multirow{2}{*}{\pass} \\ \cline{3-3}
& & Using msg.sender instead & & & & & & & \\ \hline
\multirow{2}{*}{17} & \multirow{2}{*}{116} & \tx{Block values as a proxy for time} & \multirow{2}{*}{\pass} & \multirow{2}{*}{\pass} & \multirow{2}{*}{\pass} & \multirow{2}{*}{\pass} & \multirow{2}{*}{\pass} & \multirow{2}{*}{\na} & \multirow{2}{*}{\pass} \\ \cline{3-3}
& & Using oracles instead of block number & & & & & & & \\ \hline
\multirow{2}{*}{18} & \multirow{2}{*}{117} & \tx{Signature Malleability} & \multirow{2}{*}{\na} & \multirow{2}{*}{\na} & \multirow{2}{*}{\na} & \multirow{2}{*}{\pass} & \multirow{2}{*}{\na} & \multirow{2}{*}{\na} & \multirow{2}{*}{\pass} \\ \cline{3-3}
& & Not using signed message hash & & & & & & & \\ \hline
\multirow{2}{*}{19} & \multirow{2}{*}{118} & \tx{Incorrect Constructor Name} & \multirow{2}{*}{\na} & \multirow{2}{*}{\pass} & \multirow{2}{*}{\na} & \multirow{2}{*}{\pass} & \multirow{2}{*}{\na} & \multirow{2}{*}{\na} & \multirow{2}{*}{\pass} \\ \cline{3-3}
& & Using constructor keyword & & & & & & & \\ \hline
\multirow{2}{*}{20} & \multirow{2}{*}{119} & \tx{Shadowing State Variables} & \multirow{2}{*}{\na} & \multirow{2}{*}{\na} & \multirow{2}{*}{\pass} & \multirow{2}{*}{\pass} & \multirow{2}{*}{\pass} & \multirow{2}{*}{\pass} & \multirow{2}{*}{\pass} \\ \cline{3-3}
& & Remove any variable ambiguities & & & & & & & \\ \hline
\multirow{2}{*}{21} & \multirow{2}{*}{120} & \tx{Weak Sources of Randomness from Chain Attributes} & \multirow{2}{*}{\pass} & \multirow{2}{*}{\pass} & \multirow{2}{*}{\na} & \multirow{2}{*}{\pass} & \multirow{2}{*}{\pass} & \multirow{2}{*}{\na} & \multirow{2}{*}{\pass} \\ \cline{3-3}
& & Not using block variables & & & & & & & \\ \hline
\multirow{2}{*}{22} & \multirow{2}{*}{121} & \tx{Missing Protection against Signature Replay Attacks} & \multirow{2}{*}{\na} & \multirow{2}{*}{\na} & \multirow{2}{*}{\na} & \multirow{2}{*}{\pass} & \multirow{2}{*}{\na} & \multirow{2}{*}{\na} & \multirow{2}{*}{\pass} \\ \cline{3-3}
& & Storing every message hash & & & & & & & \\ \hline
\multirow{2}{*}{23} & \multirow{2}{*}{122} & \tx{Lack of Proper Signature Verification} & \multirow{2}{*}{\na} & \multirow{2}{*}{\na} & \multirow{2}{*}{\na} & \multirow{2}{*}{\pass} & \multirow{2}{*}{\na} & \multirow{2}{*}{\na} & \multirow{2}{*}{\pass} \\ \cline{3-3}
& & Using alternate verification schemes & & & & & & & \\ \hline
\multirow{2}{*}{24} & \multirow{2}{*}{123} & \tx{Requirement Violation} & \multirow{2}{*}{\na} & \multirow{2}{*}{\pass} & \multirow{2}{*}{\pass} & \multirow{2}{*}{\pass} & \multirow{2}{*}{\na} & \multirow{2}{*}{\na} & \multirow{2}{*}{\pass} \\ \cline{3-3}
& & Allowing all valid external inputs & & & & & & & \\ \hline
\multirow{2}{*}{25} & \multirow{2}{*}{124} & \tx{Write to Arbitrary Storage Location} & \multirow{2}{*}{\na} & \multirow{2}{*}{\pass} & \multirow{2}{*}{\pass} & \multirow{2}{*}{\pass} & \multirow{2}{*}{\na} & \multirow{2}{*}{\na} & \multirow{2}{*}{\pass} \\ \cline{3-3}
& & Controlling write to sensitive storage & & & & & & & \\ \hline
\multirow{2}{*}{26} & \multirow{2}{*}{125} & \tx{Incorrect Inheritance Order} & \multirow{2}{*}{\na} & \multirow{2}{*}{\na} & \multirow{2}{*}{\na} & \multirow{2}{*}{\pass} & \multirow{2}{*}{\na} & \multirow{2}{*}{\na} & \multirow{2}{*}{\pass} \\ \cline{3-3}
& & Inheriting from more general to specific & & & & & & & \\ \hline
\multirow{2}{*}{27} & \multirow{2}{*}{126} & \tx{Insufficient Gas Griefing} & \multirow{2}{*}{\na} & \multirow{2}{*}{\pass} & \multirow{2}{*}{\na} & \multirow{2}{*}{\na} & \multirow{2}{*}{\na} & \multirow{2}{*}{\na} & \multirow{2}{*}{\pass} \\ \cline{3-3}
& & Allowing trusted forwarders & & & & & & & \\ \hline
\multirow{2}{*}{28} & \multirow{2}{*}{127} & \tx{Arbitrary Jump with Function Type Variable} & \multirow{2}{*}{\na} & \multirow{2}{*}{\pass} & \multirow{2}{*}{\pass} & \multirow{2}{*}{\pass} & \multirow{2}{*}{\na} & \multirow{2}{*}{\pass} & \multirow{2}{*}{\pass} \\ \cline{3-3}
& & Minimizing use of assembly & & & & & & & \\ \hline
\multirow{2}{*}{29} & \multirow{2}{*}{128} & \tx{DoS With Block Gas Limit} & \multirow{2}{*}{\pass} & \multirow{2}{*}{\pass} & \multirow{2}{*}{\pass} & \multirow{2}{*}{\pass} & \multirow{2}{*}{\pass} & \multirow{2}{*}{\pass} & \multirow{2}{*}{\pass} \\ \cline{3-3}
& & Avoiding loops across the entire data & & & & & & & \\ \hline
\multirow{2}{*}{30} & \multirow{2}{*}{129} & \tx{Typographical Error} & \multirow{2}{*}{\na} & \multirow{2}{*}{\na} & \multirow{2}{*}{\na} & \multirow{2}{*}{\pass} & \multirow{2}{*}{\na} & \multirow{2}{*}{\na} & \multirow{2}{*}{\pass} \\ \cline{3-3}
& & Using SafeMath & & & & & & & \\ \hline

\end{tabular}
\end{adjustbox}	
\caption{Auditing results of 7 smart contract analysis tools on \sys. \pass=Passed audit, \fp=False positive, \fail=Failed audit, Empty=Not supported audit by the tool, \info=Informational, \noSWC=Tool specific audit (No SWC registry), BP=Best practice\label{tab:result1}}
\end{table*}

% !TEX root = ../main.tex

\begin{table*}
\centering
\begin{adjustbox}{max height=10cm}
\begin{tabular}{|c|c|m{9cm}|m{5mm}|m{5mm}|m{5mm}|m{5mm}|m{5mm}|m{5mm}|m{5mm}|}

\multicolumn{3}{c}{\.} &
\headrow{EY Review} &
\headrow{Smart Check} &
\headrow{Securify} &
\headrow{MythX (Mythril)} &
\headrow{Contract Guard} &
\headrow{Slither} &
\headrow{Odin} \\ \hline

\textbf{ID} & 
\textbf{SWC} & 
\textbf{Vulnerability or best practice \newline Mitigation or recommendation} &  
\multicolumn{7}{c|}{\textbf{Tool}} \\
\hline
%			\hline\centering 1 & 100 & Function default visibility \newline Specifying function visibility & \notcovered & \passed & \notcovered & \passed & \passed & \notcovered & \passed \\
%			\hline\centering 2 & 101 & Integer Overflow and Underflow \newline Using SafeMath & \falsepos & \passed & \notcovered & \passed & \passed & \notcovered & \passed \\
%			\hline\centering 3 & 102 & Outdated Compiler Version \newline Using proper Solidity version & \passed & \passed & \passed & \passed & \passed & \passed & \failed \\
%			\hline\centering 4 & 103 & Floating Pragma \newline Locking the pragma version & \notcovered & \passed & \passed & \passed & \notcovered & \passed & \passed \\
%			\hline\centering 5 & 104 & Unchecked Call Return Value \newline Checking call() return value & \passed & \notcovered & \passed & \passed & \passed & \passed & \passed \\
%			\hline\centering 6 & 105 & Unprotected Ether Withdrawal \newline Authorizing trusted parties & \notcovered & \failed & \notcovered & \passed & \notcovered & \passed & \passed \\
%			\hline\centering 7 & 106 & Unprotected SELFDESTRUCT Instruction \newline Approving by multiple parties & \notcovered & \notcovered & \passed & \passed & \notcovered & \passed & \passed \\
%			\hline\centering 8 & 107 & Re-entrancy \newline Using CEI or Mutex & \notcovered & \passed & \falsepos & \falsepos & \falsepos & \passed & \passed \\
%			\hline\centering 9 & 108 & State variable default visibility \newline Specifying variable visibility & \passed & \failed & \passed & \passed & \passed & \notcovered & \passed \\
%			\hline\centering 10 & 109 & Uninitialized Storage Pointer \newline Initializing upon declaration & \notcovered & \notcovered & \passed & \passed & \passed & \passed & \passed \\
%			\hline\centering 11 & 110 & Assert Violation \newline Using require() statement & \notcovered & \passed & \notcovered & \passed & \notcovered & \notcovered & \passed \\
%			\hline\centering 12 & 111 & Use of Deprecated Solidity Functions \newline Using new alternatives & \notcovered & \passed & \notcovered & \passed & \passed & \passed & \passed \\
%			\hline\centering 13 & 112 & Delegatecall to untrusted callee \newline Using for trusted contracts & \notcovered & \failed & \falsepos & \passed & \passed & \passed & \passed \\
%			\hline\centering 14 & 113 & DoS with Failed Call \newline Avoid multiple external calls & \notcovered & \passed & \notcovered & \passed & \passed & \notcovered & \passed \\				
%			\hline\centering 15 & 114 & Transaction Order Dependence \newline Preventing race conditions & \falsepos & \notcovered & \passed & \passed & \notcovered & \notcovered & \passed \\
%			\hline\centering 16 & 115 & Authorization through tx.origin \newline Using msg.sender instead & \passed & \passed & \passed & \passed & \passed & \passed & \passed \\		
%			\hline\centering 17 & 116 & Block values as a proxy for time \newline Using oracles instead of block number & \passed & \passed & \passed & \passed & \passed & \notcovered & \passed \\
%			\hline\centering 18 & 117 & Signature Malleability \newline Not using signed message hash & \notcovered & \notcovered & \notcovered & \passed & \notcovered & \notcovered & \passed \\
%			\hline\centering 19 & 118 & Incorrect Constructor Name \newline Using constructor keyword & \notcovered & \passed & \notcovered & \passed & \notcovered & \notcovered & \passed \\
%			\hline\centering 20 & 119 & Shadowing State Variables \newline Remove any variable ambiguities & \notcovered & \notcovered & \passed & \passed & \passed & \passed & \passed \\
%			\hline\centering 21 & 120 & Weak Sources of Randomness from Chain Attributes \newline Not using block variables & \passed & \passed & \notcovered & \passed & \passed & \notcovered & \passed \\
%			\hline\centering 22 & 121 & Missing Protection against Signature Replay Attacks \newline Storing every message hash & \notcovered & \notcovered & \notcovered & \passed & \notcovered & \notcovered & \passed \\
%			\hline\centering 23 & 122 & Lack of Proper Signature Verification \newline Using alternate verification schemes & \notcovered & \notcovered & \notcovered & \passed & \notcovered & \notcovered & \passed \\
%			\hline\centering 24 & 123 & Requirement Violation \newline Allowing all valid external inputs & \notcovered & \passed & \passed & \passed & \notcovered & \notcovered & \passed \\
%			\hline\centering 25 & 124 & Write to Arbitrary Storage Location \newline Controlling write to sensitive storage & \notcovered & \passed & \passed & \passed & \notcovered & \notcovered & \passed \\
%			\hline\centering 26 & 125 & Incorrect Inheritance Order \newline Inheriting from more general to specific & \notcovered & \notcovered & \notcovered & \passed & \notcovered & \notcovered & \passed \\
%			\hline\centering 27 & 126 & Insufficient Gas Griefing \newline Allowing trusted forwarders & \notcovered & \passed & \notcovered & \notcovered & \notcovered & \notcovered & \passed \\
%			\hline\centering 28 & 127 & Arbitrary Jump with Function Type Variable \newline Minimizing use of assembly & \notcovered & \passed & \passed & \passed & \notcovered & \passed & \passed \\
%			\hline\centering 29 & 128 & DoS With Block Gas Limit \newline Avoiding loops across the entire data & \passed & \passed & \passed & \passed & \passed & \passed & \passed \\
%			\hline\centering 30 & 129 & Typographical Error \newline Using SafeMath & \notcovered & \notcovered & \notcovered & \passed & \notcovered & \notcovered & \passed \\
			\hline\centering 31 & 130 & Right-To-Left-Override control character (U+202E) \newline Avoiding U+202E character & \notcovered & \notcovered & \passed & \passed & \passed & \passed & \passed \\
			\hline\centering 32 & 131 & Presence of unused variables \newline Removing all unused variables & \notcovered & \passed & \passed & \notcovered & \passed & \passed & \falsepos \\
			\hline\centering 33 & 132 & Unexpected Ether balance \newline Avoiding strict Ether balance checks & \notcovered & \passed & \passed & \notcovered & \passed & \passed & \passed \\
			\hline\centering 34 & 133 & Hash Collisions With Variable Length Arguments \newline Using abi.encode() instead & \notcovered & \notcovered & \notcovered & \notcovered & \notcovered & \notcovered & \passed \\
			\hline\centering 35 & 134 & Message call with hardcoded gas amount \newline Using .call.value(...)("") & \notcovered & \falsepos & \failed & \notcovered & \passed & \notcovered & \passed \\
			\hline\centering 36 & 135 & Code With No Effects \newline Writing unit tests to verify correct behavior & \notcovered & \passed & \notcovered & \notcovered & \notcovered & \notcovered & \passed \\
			\hline\centering 37 & 136 & Unencrypted Private Data On-Chain \newline Storing private data off-chain & \notcovered & \notcovered & \notcovered & \notcovered & \notcovered & \notcovered & \passed \\		
			\hline\centering 38 & \BP & ERC20 compliance \newline Implementing 6 functions and 2 events & \passed & \passed & \passed & \notcovered & \passed & \passed & \notcovered \\ 
			\hline\centering 39 & \BP & Number of external functions \newline Minimizing external functions & \passed & \passed & \passed & \notcovered & \notcovered & \notcovered & \notcovered \\ 
			\hline\centering 40 & \BP & Token decimal \newline Adding a token decimal declaration & \passed & \notcovered & \notcovered & \notcovered & \notcovered & \notcovered & \notcovered \\
			\hline\centering 41 & \BP & Token name \newline Adding a token name variable & \passed & \notcovered & \notcovered & \notcovered & \notcovered & \notcovered & \notcovered \\
			\hline\centering 42 & \BP & Token symbol \newline Adding a token symbol variable & \passed & \notcovered & \notcovered & \notcovered & \notcovered & \notcovered & \notcovered \\
			\hline\centering 43 & \noSWC & Allowance decreases upon transfer \newline Decreasing allowance in transferFrom() & \failed & \notcovered & \notcovered & \notcovered & \notcovered & \notcovered & \notcovered \\
			\hline\centering 44 & \noSWC & Allowance function returns an accurate value \newline Returning only value from the mapping & \passed & \notcovered & \notcovered & \notcovered & \notcovered & \notcovered & \notcovered \\
			\hline\centering 45 & \BP & Allowance spending is possible \newline Ability of token transfer by transferFrom() & \passed & \notcovered & \notcovered & \notcovered & \notcovered & \notcovered & \notcovered  \\
			\hline\centering 46 & \BP & The Approval event is correctly logged \newline Emitting Approval event & \passed & \notcovered & \notcovered & \notcovered & \notcovered & \notcovered & \notcovered \\
			\hline\centering 47 & \noSWC & It is possible to cancel an existing allowance \newline Possibility of setting allowance to 0 & \passed & \passed & \notcovered & \notcovered & \notcovered & \notcovered & \notcovered \\
			\hline\centering 48 & \BP & The decreaseAllowance definition follows the standard \newline Defining decreaseAllowance function & \falsepos & \notcovered & \notcovered & \notcovered & \notcovered & \notcovered & \notcovered \\
			\hline\centering 49 & \BP & The increaseAllowance definition follows the standard \newline Defining increaseAllowance function & \falsepos & \notcovered & \notcovered & \notcovered & \notcovered & \notcovered & \notcovered \\
			\hline\centering 50 & \noSWC & A transfer with an insufficient amount is reverted \newline Checking balances in transfer() & \passed & \notcovered & \notcovered & \notcovered & \notcovered & \passed & \notcovered \\
			\hline\centering 51 & \BP & Uninitialized state variables \newline Initializing all the variables & \passed & \passed & \notcovered & \notcovered & \passed & \passed & \notcovered \\
			\hline\centering 52 & \noSWC & Upon sending funds, the sender's balance is updated \newline Updating balances in transfer() & \passed & \notcovered & \notcovered & \notcovered & \notcovered & \notcovered & \notcovered \\
			\hline\centering 53 & \noSWC & The Transfer event correctly logged \newline Emitting Transfer event & \passed & \notcovered & \notcovered & \notcovered & \notcovered & \notcovered & \notcovered \\
			\hline\centering 54 & \BP & Transfer to the burn address is reverted \newline Checking transfer to 0x0 & \passed & \notcovered & \notcovered & \notcovered & \notcovered & \notcovered & \notcovered \\
			\hline\centering 55 & \noSWC & Transfer an amount that is greater than the allowance \newline Checking in transferFrom() & \passed & \notcovered & \notcovered & \notcovered & \notcovered & \notcovered & \notcovered \\
			\hline\centering 56 & \BP & Emitting event when state changes \newline Emitting Change event & \failed & \notcovered & \notcovered & \notcovered & \notcovered & \notcovered & \notcovered \\
			\hline\centering 57 & \BP & Source code is decentralized \newline Not using hard-coded addresses & \passed & \passed & \notcovered & \notcovered & \notcovered & \notcovered & \notcovered \\
			\hline\centering 58 & \noSWC & Risk of short address attack is minimized \newline Using recent Solidity version & \passed & \notcovered & \notcovered & \notcovered & \passed & \notcovered & \notcovered \\
			\hline\centering 59 & \noSWC & Function names are unique \newline No function overloading & \falsepos & \notcovered & \notcovered & \notcovered & \notcovered & \passed & \notcovered \\
			\hline\centering 60 & \BP & Funds can be held only by user-controlled wallets \newline Checking for address code & \failed & \notcovered & \notcovered & \notcovered & \notcovered & \notcovered & \notcovered \\
%			\hline\centering 61 & \BP & Code logic is simple to understand \newline Avoiding code nesting & \passed & \passed & \notcovered & \notcovered & \notcovered & \notcovered & \notcovered \\
%			\hline\centering 62 & \BP & All functions are documented \newline Using NatSpec format & \passed & \notcovered & \notcovered &\notcovered  & \notcovered & \notcovered & \notcovered \\
%			\hline\centering 63 & \BP & Using only high-level programming language \newline Not using inline-assembly codes & \passed & \passed & \passed & \notcovered & \passed & \passed & \notcovered \\
%			\hline\centering 64 & \BP & Acceptable gas cost of the approve() function \newline Checking for maximum 50000 gas & \failed & \notcovered & \notcovered & \notcovered & \notcovered & \notcovered & \notcovered \\
%			\hline\centering 65 & \BP & Acceptable gas cost of the transfer() function \newline Checking for maximum 60000 gas & \failed & \notcovered & \notcovered & \notcovered & \notcovered & \notcovered & \notcovered \\
%			\hline\centering 66 & \BP & Use of "Pull over Push" efficiency pattern \newline Allowing user to pull the funds & \passed & \passed & \notcovered & \notcovered & \notcovered & \notcovered & \notcovered \\
%			\hline\centering 67 & \BP & Use of unindexed arguments \newline Using indexed events' arguments & \notcovered & \passed & \notcovered & \notcovered & \passed & \passed & \notcovered \\
%			\hline\centering 68 & \noSWC & Using miner controlled variables \newline Avoiding now, block.timestamp, etc & \passed & \passed & \passed & \passed & \passed & \passed & \notcovered \\
%			\hline\centering 69 & \noSWC & Use of return in constructor \newline Not using return in contract's constructor & \notcovered & \passed & \notcovered & \notcovered & \notcovered & \notcovered & \notcovered \\
%			\hline\centering 70 & \noSWC & Throwing exceptions in transfer() and transferFrom() \newline Returning true after successful execution & \notcovered & \passed & \notcovered & \notcovered & \notcovered & \passed & \notcovered \\
%			\hline\centering 71 & \BP & Locked money \newline Implement a withdraw function or reject payments & \notcovered &	\passed & \notcovered & \notcovered & \notcovered & \passed & \notcovered \\
%			\hline\centering 72 & \BP & Malicious libraries \newline Not using modifiable third-party libraries & \notcovered & \passed & \notcovered & \notcovered & \notcovered & \notcovered & \notcovered \\
%			\hline\centering 73 & \BP & Payable fallback function \newline Adding fallback() function to receive Ether & \notcovered & \passed & \notcovered & \notcovered & \passed & \notcovered & \notcovered \\
%			\hline\centering 74 & \BP & Prefer external to public visibility level \newline Replacing public with external if not used locally & \notcovered & \passed & \notcovered & \notcovered & \notcovered & \passed & \notcovered \\
%			\hline\centering 75 & \noSWC & Call with hard-coded gas amount (EIP1884) \newline Not using transfer() or send() functions & \notcovered & \failed & \notcovered & \passed & \notcovered & \notcovered & \notcovered \\
%			\hline\centering 76 & \BP & Error information in revert condition \newline Adding error description & \notcovered & \notcovered & \notcovered & \notcovered & \passed & \notcovered & \notcovered \\
%			\hline\centering 77 & \BP & Freezing Ether \newline Adding functions to send Ether out & \notcovered & \passed & \notcovered & \notcovered & \passed & \notcovered & \notcovered \\
%			\hline\centering 78 & \BP & Complex Fallback \newline Logging operations in the fallback function & \notcovered & \notcovered & \notcovered & \notcovered & \passed & \notcovered & \notcovered \\
%			\hline\centering 79 & \BP & Function Order \newline Following fallback, external, etc & \notcovered & \notcovered & \notcovered & \notcovered & \passed & \notcovered & \notcovered \\
%			\hline\centering 80 & \BP & Visibility Modifier Order \newline Specifying visibility first and before modifiers & \notcovered & \notcovered & \notcovered & \notcovered & \passed & \notcovered & \notcovered \\
%			\hline\centering 81 & \BP & Non-initialized return value \newline Not specifying return for functions without output & \notcovered & \passed & \notcovered & \notcovered & \passed & \notcovered & \notcovered \\
%			\hline\centering 82 & \noSWC & Tautology or contradiction \newline Fixing comparison that are always true or false & \notcovered & \notcovered & \notcovered & \notcovered & \notcovered & \passed & \notcovered \\
%			\hline\centering 83 & \noSWC & Divide before multiply \newline Ordering multiplication prior division & \notcovered & \notcovered & \notcovered & \notcovered & \notcovered & \passed & \notcovered \\
%			\hline\centering 84 & \noSWC & Unchecked Send \newline Ensure that the return value of send() is checked & \notcovered & \notcovered & \notcovered & \notcovered & \notcovered & \passed & \notcovered \\
%			\hline\centering 85 & \BP & Builtin Symbol Shadowing \newline Renaming variables & \notcovered & \notcovered & \notcovered & \notcovered & \notcovered & \passed & \notcovered \\
%			\hline\centering 86 & \BP & Low level calls \newline Checking successful return value from call() & \notcovered & \notcovered & \notcovered & \notcovered & \notcovered & \failed & \notcovered \\
%			\hline\centering 87 & \BP & Conformance to naming conventions \newline Following the Solidity naming convention & \notcovered & \notcovered & \notcovered & \notcovered & \notcovered & \passed & \notcovered \\
%			\hline\centering 88 & \BP & Too many digits \newline Using scientific notation & \notcovered & \notcovered & \notcovered & \notcovered & \notcovered & \passed & \notcovered \\
%			\hline\centering 89 & \BP & State variables that could be declared constant \newline Adding constant attribute & \notcovered & \notcovered & \notcovered & \notcovered & \notcovered & \passed & \notcovered \\

\hline
\end{tabular}
\end{adjustbox}	
\caption{Continuation of Table~\ref{tab:result}.\label{tab:result2}}
\end{table*}

% !TEX root = ../main.tex

\begin{table*}
\centering
\begin{adjustbox}{max height=10cm}
\begin{tabular}{|P{3mm}|P{7mm}|m{95mm}|P{7mm}|P{7mm}|P{7mm}|P{7mm}|P{7mm}|P{7mm}|P{7mm}|}

\multicolumn{3}{c}{\.} &
\headrow{EY Token Review} &
\headrow{Smart Check} &
\headrow{Securify} &
\headrow{MythX (Mythril)} &
\headrow{Contract Guard} &
\headrow{Slither} &
\headrow{Odin} \\ \hline

\rcl\ccl & \ccl & \tx{Vulnerability or best practice} & \multicolumn{7}{c|}{\ccl} \\ \cline{3-3}
\rcl\multirow{-2}{*}{\ccl\tx{ID}} & \multirow{-2}{*}{\ccl\tx{SWC}} & Mitigation or recommendation & \multicolumn{7}{c|}{\multirow{-2}{*}{\ccl\tx{Security tools}}} \\ \hline

\multirow{2}{*}{55} & \multirow{2}{*}{\BP} & \tx{The decreaseAllowance definition follows the standard} & \multirow{2}{*}{\pass} & \multirow{2}{*}{\na} & \multirow{2}{*}{\na} & \multirow{2}{*}{\na} & \multirow{2}{*}{\na} & \multirow{2}{*}{\na} & \multirow{2}{*}{\na} \\ \cline{3-3} & & Defining decreaseAllowance input and output variables as standard & & & & & & & \\ \hline
\multirow{2}{*}{56} & \multirow{2}{*}{\BP} & \tx{The increaseAllowance definition follows the standard} & \multirow{2}{*}{\pass} & \multirow{2}{*}{\na} & \multirow{2}{*}{\na} & \multirow{2}{*}{\na} & \multirow{2}{*}{\na} & \multirow{2}{*}{\na} & \multirow{2}{*}{\na} \\ \cline{3-3} & & Defining increaseAllowance input and output variables as standard & & & & & & & \\ \hline
\multirow{2}{*}{57} & \multirow{2}{*}{\BP} & \tx{Minimize attack surface} & \multirow{2}{*}{\pass} & \multirow{2}{*}{\pass} & \multirow{2}{*}{\pass} & \multirow{2}{*}{\na} & \multirow{2}{*}{\na} & \multirow{2}{*}{\na} & \multirow{2}{*}{\na} \\ \cline{3-3} & & Checking whether all the external functions are necessary or not & & & & & & & \\ \hline
\multirow{2}{*}{58} & \multirow{2}{*}{\BP} & \tx{Transfer to the burn address is reverted} & \multirow{2}{*}{\pass} & \multirow{2}{*}{\na} & \multirow{2}{*}{\na} & \multirow{2}{*}{\na} & \multirow{2}{*}{\na} & \multirow{2}{*}{\na} & \multirow{2}{*}{\na} \\ \cline{3-3} & & Reverting transfer to 0x0 due to risk of total supply reduction & & & & & & & \\ \hline
\multirow{2}{*}{59} & \multirow{2}{*}{\BP} & \tx{Source code is decentralized} & \multirow{2}{*}{\pass} & \multirow{2}{*}{\pass} & \multirow{2}{*}{\na} & \multirow{2}{*}{\na} & \multirow{2}{*}{\na} & \multirow{2}{*}{\na} & \multirow{2}{*}{\na} \\ \cline{3-3} & & Not using hard-coded addresses in the code & & & & & & & \\ \hline
\multirow{2}{*}{60} & \multirow{2}{*}{\BP} & \tx{Funds can be held only by user-controlled wallets} & \multirow{2}{*}{\info} & \multirow{2}{*}{\na} & \multirow{2}{*}{\na} & \multirow{2}{*}{\na} & \multirow{2}{*}{\na} & \multirow{2}{*}{\na} & \multirow{2}{*}{\na} \\ \cline{3-3} & & Transferring tokens to users to avoid creating a secondary market & & & & & & & \\ \hline
\multirow{2}{*}{61} & \multirow{2}{*}{\BP} & \tx{Code logic is simple to understand} & \multirow{2}{*}{\pass} & \multirow{2}{*}{\pass} & \multirow{2}{*}{\na} & \multirow{2}{*}{\na} & \multirow{2}{*}{\na} & \multirow{2}{*}{\na} & \multirow{2}{*}{\na} \\ \cline{3-3} & & Avoiding code nesting which makes the code less intuitive & & & & & & & \\ \hline
\multirow{2}{*}{62} & \multirow{2}{*}{\BP} & \tx{All functions are documented} & \multirow{2}{*}{\pass} & \multirow{2}{*}{\na} & \multirow{2}{*}{\na} &\multirow{2}{*}{\na} & \multirow{2}{*}{\na} & \multirow{2}{*}{\na} & \multirow{2}{*}{\na} \\ \cline{3-3} & & Using NatSpec format to explain expected behavior of functions & & & & & & & \\ \hline
\multirow{2}{*}{63} & \multirow{2}{*}{\BP} & \tx{The Approval event is correctly logged} & \multirow{2}{*}{\pass} & \multirow{2}{*}{\na} & \multirow{2}{*}{\na} & \multirow{2}{*}{\na} & \multirow{2}{*}{\na} & \multirow{2}{*}{\na} & \multirow{2}{*}{\na} \\ \cline{3-3} & & Emitting Approval event in the approve() method & & & & & & & \\ \hline
\multirow{2}{*}{64} & \multirow{2}{*}{\BP} & \tx{Acceptable gas cost of the approve() function} & \multirow{2}{*}{\info} & \multirow{2}{*}{\na} & \multirow{2}{*}{\na} & \multirow{2}{*}{\na} & \multirow{2}{*}{\na} & \multirow{2}{*}{\na} & \multirow{2}{*}{\na} \\ \cline{3-3} & & Checking for maximum 50000 gas cost when executing the approve() & & & & & & & \\ \hline
\multirow{2}{*}{65} & \multirow{2}{*}{\BP} & \tx{Acceptable gas cost of the transfer() function} & \multirow{2}{*}{\info} & \multirow{2}{*}{\na} & \multirow{2}{*}{\na} & \multirow{2}{*}{\na} & \multirow{2}{*}{\na} & \multirow{2}{*}{\na} & \multirow{2}{*}{\na} \\ \cline{3-3} & & Checking for maximum 60000 gas cost when executing the transfer() & & & & & & & \\ \hline
\multirow{2}{*}{66} & \multirow{2}{*}{\BP} & \tx{Emitting event when state changes} & \multirow{2}{*}{\pass} & \multirow{2}{*}{\na} & \multirow{2}{*}{\na} & \multirow{2}{*}{\na} & \multirow{2}{*}{\na} & \multirow{2}{*}{\na} & \multirow{2}{*}{\na} \\ \cline{3-3} & & Emitting Change event when changing state variable values & & & & & & & \\ \hline
\multirow{2}{*}{67} & \multirow{2}{*}{\BP} & \tx{Use of unindexed arguments} & \multirow{2}{*}{\na} & \multirow{2}{*}{\pass} & \multirow{2}{*}{\na} & \multirow{2}{*}{\na} & \multirow{2}{*}{\pass} & \multirow{2}{*}{\pass} & \multirow{2}{*}{\na} \\ \cline{3-3} & & Using indexed arguments to facilitate external tools log searching & & & & & & & \\ \hline
\multirow{2}{*}{68} & \multirow{2}{*}{\BP} & \tx{\erc compliance} & \multirow{2}{*}{\pass} & \multirow{2}{*}{\pass} & \multirow{2}{*}{\pass} & \multirow{2}{*}{\na} & \multirow{2}{*}{\pass} & \multirow{2}{*}{\pass} & \multirow{2}{*}{\na} \\ \cline{3-3} & & Implementing all 6 functions and 2 events as specified in EIP-20 & & & & & & & \\ \hline
\multirow{2}{*}{69} & \multirow{2}{*}{\BP} & \tx{Conformance to naming conventions} & \multirow{2}{*}{\na} & \multirow{2}{*}{\na} & \multirow{2}{*}{\na} & \multirow{2}{*}{\na} & \multirow{2}{*}{\na} & \multirow{2}{*}{\pass} & \multirow{2}{*}{\na} \\ \cline{3-3} & & Following the Solidity naming convention to avoid confusion & & & & & & & \\ \hline
\multirow{2}{*}{70} & \multirow{2}{*}{\BP} & \tx{Token decimal} & \multirow{2}{*}{\pass} & \multirow{2}{*}{\na} & \multirow{2}{*}{\na} & \multirow{2}{*}{\na} & \multirow{2}{*}{\na} & \multirow{2}{*}{\na} & \multirow{2}{*}{\na} \\ \cline{3-3} & & Declaring token decimal for external apps when displaying balances & & & & & & & \\ \hline
\multirow{2}{*}{71} & \multirow{2}{*}{\BP} & \tx{Locked money (Freezing ETH)} & \multirow{2}{*}{\na} &\multirow{2}{*}{\pass} & \multirow{2}{*}{\na} & \multirow{2}{*}{\na} & \multirow{2}{*}{\pass} & \multirow{2}{*}{\pass} & \multirow{2}{*}{\na} \\ \cline{3-3} & & Implementing withdraw/reject functions to avoid ETH lost & & & & & & & \\ \hline
\multirow{2}{*}{72} & \multirow{2}{*}{\BP} & \tx{Malicious libraries} & \multirow{2}{*}{\na} & \multirow{2}{*}{\pass} & \multirow{2}{*}{\na} & \multirow{2}{*}{\na} & \multirow{2}{*}{\na} & \multirow{2}{*}{\na} & \multirow{2}{*}{\na} \\ \cline{3-3} & & Not using modifiable third-party libraries & & & & & & & \\ \hline
\multirow{2}{*}{73} & \multirow{2}{*}{\BP} & \tx{Payable fallback function} & \multirow{2}{*}{\na} & \multirow{2}{*}{\pass} & \multirow{2}{*}{\na} & \multirow{2}{*}{\na} & \multirow{2}{*}{\pass} & \multirow{2}{*}{\na} & \multirow{2}{*}{\na} \\ \cline{3-3} & & Adding either fallback() or receive() function to receive ETH & & & & & & & \\ \hline
F\multirow{2}{*}{74} & \multirow{2}{*}{\BP} & \tx{Prefer external to public visibility level} & \multirow{2}{*}{\na} & \multirow{2}{*}{\pass} & \multirow{2}{*}{\na} & \multirow{2}{*}{\na} & \multirow{2}{*}{\na} & \multirow{2}{*}{\pass} & \multirow{2}{*}{\na} \\ \cline{3-3} & & Improving the performance by replacing public with external & & & & & & & \\ \hline
\multirow{2}{*}{75} & \multirow{2}{*}{\BP} & \tx{Token name} & \multirow{2}{*}{\pass} & \multirow{2}{*}{\na} & \multirow{2}{*}{\na} & \multirow{2}{*}{\na} & \multirow{2}{*}{\na} & \multirow{2}{*}{\na} & \multirow{2}{*}{\na} \\ \cline{3-3} & & Adding a token name variable for external apps & & & & & & & \\ \hline
\multirow{2}{*}{76} & \multirow{2}{*}{\BP} & \tx{Error information in revert condition} & \multirow{2}{*}{\na} & \multirow{2}{*}{\na} & \multirow{2}{*}{\na} & \multirow{2}{*}{\na} & \multirow{2}{*}{\pass} & \multirow{2}{*}{\na} & \multirow{2}{*}{\na} \\ \cline{3-3} & & Adding error description in require()/revert() to clarify the reason & & & & & & & \\ \hline
\multirow{2}{*}{77} & \multirow{2}{*}{\BP} & \tx{Complex Fallback} & \multirow{2}{*}{\na} & \multirow{2}{*}{\na} & \multirow{2}{*}{\na} & \multirow{2}{*}{\na} & \multirow{2}{*}{\pass} & \multirow{2}{*}{\na} & \multirow{2}{*}{\na} \\ \cline{3-3} & & Logging operations in the fallback() to avoid complex operations & & & & & & & \\ \hline
\multirow{2}{*}{78} & \multirow{2}{*}{\BP} & \tx{Function Order} & \multirow{2}{*}{\na} & \multirow{2}{*}{\na} & \multirow{2}{*}{\na} & \multirow{2}{*}{\na} & \multirow{2}{*}{\pass} & \multirow{2}{*}{\na} & \multirow{2}{*}{\na} \\ \cline{3-3} & & Following fallback, external, public, internal and private order & & & & & & & \\ \hline
\multirow{2}{*}{79} & \multirow{2}{*}{\BP} & \tx{Visibility Modifier Order} & \multirow{2}{*}{\na} & \multirow{2}{*}{\na} & \multirow{2}{*}{\na} & \multirow{2}{*}{\na} & \multirow{2}{*}{\na} & \multirow{2}{*}{\pass} & \multirow{2}{*}{\na} \\ \cline{3-3} & & Specifying visibility first and before modifiers in functions & & & & & & & \\ \hline
\multirow{2}{*}{80} & \multirow{2}{*}{\BP} & \tx{Non-initialized return value} & \multirow{2}{*}{\na} & \multirow{2}{*}{\pass} & \multirow{2}{*}{\na} & \multirow{2}{*}{\na} & \multirow{2}{*}{\pass} & \multirow{2}{*}{\na} & \multirow{2}{*}{\na} \\ \cline{3-3} & & Not specifying return for functions without output & & & & & & & \\ \hline
\multirow{2}{*}{81} & \multirow{2}{*}{\BP} & \tx{Token symbol} & \multirow{2}{*}{\pass} & \multirow{2}{*}{\na} & \multirow{2}{*}{\na} & \multirow{2}{*}{\na} & \multirow{2}{*}{\na} & \multirow{2}{*}{\na} & \multirow{2}{*}{\na} \\ \cline{3-3} & & Adding token symbol variable for usage of external apps & & & & & & & \\ \hline
\multirow{2}{*}{82} & \multirow{2}{*}{\BP} & \tx{Allowance spending is possible} & \multirow{2}{*}{\pass} & \multirow{2}{*}{\na} & \multirow{2}{*}{\na} & \multirow{2}{*}{\na} & \multirow{2}{*}{\na} & \multirow{2}{*}{\na} & \multirow{2}{*}{\na} \\ \cline{3-3} & & Ability of token transfer by transferFrom() to transfer tokens on behalf of another usercalc & & & & & & & \\ \hline

\hline
\rcl
\multicolumn{3}{|c|}{\ccl\tx{\begin{tabular}[c]{@{}c@{}}\prct success rate in performed audits by considering\\ 'False Positives' and 'Informational' checks as 'Passed' \\ (More details in section~\ref{section:tools}) \end{tabular}}}  & 100\% & 100\% & 100\% & 100\% & 100\% & 100\% & 97\% \\ \hline


\end{tabular}
\end{adjustbox}	
\caption{Continuation of Table~\ref{tab:result2}.\label{tab:result3}}
\end{table*}


We used a variety of code audit tools on \sys to validate the code and also to illuminate the completeness and error-rate of such tools on one specific use-case (similar work studies in less depth a variety of use-cases~\cite{AuditTools}). We did not adapt older tools that support lower versions of the Solidity compiler (\eg Oyente). The following seven tools are all publicly available.

%\begin{samepage}
\begin{enumerate}
	\item EY Review Tool \footnote{\url{https://review-tool.blockchain.ey.com}} by Ernst \& Young Global Limited.
	\item SmartCheck\footnote{\url{https://tool.smartdec.net}} by SmartDec.
	\item Securify\footnote{\url{https://securify.chainsecurity.com}} by ChainSecurity.
	\item ContractGuard\footnote{\url{https://contract.guardstrike.com}} by GuardStrike.
	\item MythX\footnote{\url{https://mythx.io}} by ConsenSys.
	\item Slither Analyzer\footnote{\url{https://github.com/crytic/slither}} by Crytic.
	\item Odin\footnote{\url{https://odin.sooho.io/}} by Sooho.
\end{enumerate}
%\end{samepage}

A total of \num audits have been conducted by these auditing tools. Audits include best practices {\chg in addition to} security vulnerabilities. The results are summarized in Tables~\ref{tab:result1}--\ref{tab:result3} and sorted by Smart Contract Weakness Classification (SWC)\footnote{\url{https://swcregistry.io/}}). Knowledge-base of each tool is used to map audits to the corresponding SWC registry \cite{SECURIFYGIT,SMARTCHECK,MythX,ContractGuard,SlitherDoc}. Since each tool employs different methodology to analyze smart contracts (\eg comparing with violation patterns, applying a set of rules, using static analysis, \etc), there are false positives {\chg to manually check}. The following are some examples {\chg that we have not taken into account in calculating the success rate}:
 
\begin{itemize}
	\item \textit{MythX} detects \textit{Re-entrancy attack} in the \textit{noReentrancy} modifier. In Solidity, modifiers {\chg are not like functions. They are} used to add features or apply some restriction on functions~\cite{SolidityModifer}. Using modifiers is a known technique to implement Mutex and mitigate the attack~\cite{ReentrancyGuard}. This is a false positive and note that other tools have not identified the attack {\chg in modifiers}.

	\item {\chg\textit{ContractGuard} flags \textit{Re-entrancy attack} in \texttt{transfer()} function while both CEI and Mutex are implemented.}

	\item \textit{Slither} detects two \textit{low level call} vulnerabilities\cite{SlitherSetup}. This is due to use of \texttt{call.value()} that is recommend way of transferring ETH after \textit{Istanbul} hard-fork (EIP-1884).	Therefore, adapting analyzers to new standards can improve accuracy of the security checks.

	\item \textit{SmartCheck} recommends not using \texttt{SafeMath} and {\chg check explicitly where overflows might be occurred. We consider this failed audit as false possible whereas utilizing \texttt{SafeMath} is a known technique to mitigate over/under flows. It also flags \textit{using private modifier} as vulnerability by mentioning} ``miners have access to all contracts' data and developers must account for the lack of privacy in Ethereum''. However private visibility in Solidity concerns object oriented inheritance not confidentiality. The tool also warns against \texttt{approve()} in \erc due to front-running attacks (see above). Despite EIP-1884, the tool still recommends using of \texttt{transfer()} method with stipend of 2300 gas. {\chg There are other false positives such as SWC-105 and SWC-112 that are not detected by other tools.}

	\item \textit{Securify} detects \textit{Re-entrancy} attack due to unrestricted writes in \texttt{noReentrancy} modifier~\cite{SECURIFY}. Modifiers are not accessible by users and are the recommended approach to prevent the attack. {\chg It also flags \textit{Delegatecall to Untrusted Callee (SWC-112)} while there is no usage of \texttt{delegatecall()} in the code. It might be due to use of \texttt{SafeMath} library which is an embedded library. In Solidity, embedded libraries are called by JUMP commands instead of \texttt{delegatecall()}. Therefore, excluding embedded libraries from this check might improve accuracy of the tool. Similar to \textit{SmartCheck}, it still recommends to use \texttt{transfer()} method instead of \texttt{call.value()}}.

	\item \textit{EY token review} considers \texttt{decreaseAllowance} and \texttt{increaseAllowance} as standard \erc functions and if not implemented, recognizes the code as vulnerable to a \textit{front-running} attack. These two functions are not defined in the \erc standard~\cite{ERC20Std} and considered only by this tool as mandatory functions. There are other methods to prevent the attack while adhering \erc specifications (see Rahimian \etal for a full paper on this attack and the basis of the mitigation in \sys~\cite{ERC20MWA}). The tool also falsely detects the \textit{Overflow} attack, mitigated through \texttt{SafeMath}. Another identified issue is \textit{Funds can be held only by user-controlled wallets}. The tool warns agains any token transfer to Ethereum addresses that belong to smart contracts. However, interacting with \erc token by other smart contracts was one of the main motivations of the \erc standard. It also checks for maximum 50000 gas in \texttt{approve()} and 60000 in \texttt{transfer()} method. We could not find corresponding SWC registry or standard recommendation on these limitations {\chg and therefore consider them as informational.}

	\item \textit{Odin} raises \textit{Outdated compiler version} issue due to locking solidity version to 0.5.11. We have used this version due to its compatibility with other auditing tools. Furthermore, other tools have not identified such an issue {\chg and we therefore consider it as informational}.
\end{itemize}

After manually overriding the false positives, the average percentage of passed checks for \sys reaches to \prct. {\chg This is because of \textit{Odin} which enforces higher versions of Solidity. To pass this final check and reach to 100\% success rate across all tools, we prepared the same code in Solidity version 0.7.1\footnote{\url{https://bit.ly/33wDENx}}, however, it has the risk of not supporting by most of audit tools. Developers may need to check for vulnerabilities manually.}
