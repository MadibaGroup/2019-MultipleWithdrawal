% !TEX root = ../main.tex

\section{Introduction}
\label{sect:introduction}

The Ethereum blockchain project was launched in 2014 by announcing Ether (ETH) as its protocol-level cryptocurrency \cite{EthGit,EIP150}. Ethereum allows users to build and deploy decentralized applications (DApps), or smart contracts, that can accept and use ETH. Many DApps also issue their own custom tokens with a variety of intents, including tokens as: financial products, in-house currencies, voting rights for DApp governance, valuable assets, crypto-collectables, \etc. To encourage interoperability with other DApps and web apps (exchanges, wallets, \etc), the Ethereum community accepted a popular token standard (for non-fungible tokens) called ERC20\footnote{\url{https://eips.ethereum.org/EIPS/eip-20}}. While a numerous ERC20 extensions or replacements have been proposed, ERC20 remains prominent. From about 2.5M\footnote{[2020-05-03] \url{https://reports.aleth.io}} smart contracts on the Ethereum network, 260K\footnote{[2020-05-03] \url{https://etherscan.io/tokens}} are ERC20 tokens.

The development of smart contracts has been proven to be error-prone, and as a result, smart contracts are often riddled with security vulnerabilities. An early study in 2016 found that 45\% of smart contracts at that time had vulnerablities~\cite{MakSm}. In the ensuing years, a concentration on security was made by the community, which includes the development of security auditing tools (typically using static analysis). ERC20 token security is particularly important given that many tokens have considerable value, some exceeding the value of ETH itself (\eg PAX Gold, MKR and XIN). 

As tokens can be held by commercial firms, in addition to individuals, and firms need audited financial statements in certain circumstances, the correctness of the contract issuing the tokens is now in the purview of professional auditors. One tool we examine, EY Review, is from a `big-four' auditing firm. 

\paragraph{Contributions.} Similar to any new technology, different layers of Ethereum (\eg Application, Contract, \etc) expose security vulnerabilities that have collectively caused more than US\$100M in financial losses~\cite{DAO1,PeckShield,PartiyMultiSig,MyEthWallet,ParityFirstHack,ParitySecondHack}. This motivates us to (i) comprehensively study all known vulnerabilities in ERC20 token contracts, knowledge that we systemize\footnote{Note to reviewers: we debated if our paper is an SoK or not but decided because of (iv), it is not a pure SoK. We are open to having it appear in either category.} into a set of 89 distinct vulnerabilities; (ii) review the completeness of auditing tools in detecting these vulnerabilities to establish how reliable an audit is that uses one of these tools, (iii) examine the practicality of our work in the context of the top ten ERC20 tokens, and (iv) use this research to provide a new secure implementation of the ERC20 interface that is freely available and open source. While we focus on ERC20, many ERC20 proposed replacements (\ie ERC 223~\cite{Ref20}, 667~\cite{Ref21}, 721~\cite{Ref22}, 777~\cite{Ref23}, 827~\cite{Ref24}, 1155~\cite{Ref25}, 1377~\cite{Ref26}) either fully subsume ERC20 functionality (\ie they extend the ERC20 interface) or they overlap considerably. 