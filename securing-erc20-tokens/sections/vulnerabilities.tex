% !TEX root = ../main.tex

\section{ERC20 security vulnerabilities}\label{sec:vul}
ERC20 tokens are subset of smart contracts and vulnerable in a similar way. We therefore examine attack vector and broader impact of smart contract vulnerabilities \cite{SolidtySecBlog}, \cite{EthSecServ}, \cite{SoliditySecCon}, \cite{ConsensysSecCon}, \cite{LandoKL} to check their applicability on ERC20 tokens. For each vulnerability, we (i) briefly explain technical details, (ii) ability to affect ERC20 tokens, (iii) discuss mitigation technique. We ultimately put all of the mitigation techniques together and propose a secure ERC20 code that is not vulnerable to any of discussed threats (See Section~\ref{sec:proposal}).

\begin{figure}[t!]
	\centering
	\includegraphics[width=1.0\linewidth]{figures/blockchain.png}
	\caption{Architecture of Ethereum blockchain, including interactive environment (\ie Application layer). ERC20 tokens falls under \textit{Smart Contracts} category in \textit{Contract Layer}. The security vulnerabilities of smart contracts can be extended to tokens as well. There might be also security vulnerabilities in other layers that we focus only on ERC20 tokens.}\label{fig:blockchain}
\end{figure}

Among the layers of Ethereum blockchain, our focus is on the \textit{Contract layer} in which smart contracts are executed (See figure~\ref{fig:blockchain}). The presence of security vulnerability in supplementary layers affect the entire Ethereum blockchain, not necessarily ERC20 tokens. Therefore, vulnerabilities in other layers are assumed to be out of the scope (\eg \textit{Indistinguishable chains} at Data layer, \textit{51\% hashrate} at Consensus layer, \textit{Unlimited nodes creation} at Network layer and \textit{Web3.js Arbitrary File Write} at Application layer). Moreover, due to the use of the recent version of Solidity compiler, we do not discuss the vulnerabilities identified in the outdated compiler versions, for example:
\begin{itemize}
	\item \textit{Constructor name ambiguity} in versions before 0.4.22.
	\item \textit{Uninitialized storage pointer} in versions before 0.5.0.
	\item \textit{Function default visibility} in versions before 0.5.0
	\item \textit{Typographical error} in versions before 0.5.8.
	\item \textit{Deprecated solidity functions} in versions before 0.4.25.
	\item \textit{Assert Violation} in versions before 0.4.10.
	\item \textit{Under-priced DoS attack} before EIP-150 \& EIP-1884.
\end{itemize}

\subsubsection*{2.1. Arithmetic Over/Under Flows}
It is well known issue in many programming languages called \textit{integer overflow}\footnote{\url{http://bit.ly/3cJDqX6}}. It was exploited in April, 2018 and some exchanges\footnote{OKEx, Poloniex, HitBTC and Huobi Pro} had suspended deposits and withdrawals of all tokens, especially for Beauty Ecosystem Coin (BEC\footnote{\url{http://bit.ly/2TIartO}}) that was targeted by this exploit. Although BEC developers had considered most of the security measurements, only line 261\footnote{\url{http://bit.ly/38BwcRI}} was vulnerable\cite{PeckShield}. The attacker was able to pass a combination of input values to transfer large amount of tokens\cite{Overflow}. It was even larger than the initial supply of the token, allowing the attacker to take control of token finance and manipulate the price. In Ethereum, integer overflow does not throw exception at runtime. This is by design and can be prevented by using \texttt{SafeMath}\footnote{\url{http://bit.ly/2VYuoPU}} library where in \texttt{a+b} will be replaced by \texttt{a.add(b)} and throws an exception in case of arithmetic overflow. This library is offered by OpenZeppelin\footnote{\url{http://bit.ly/2Tx8DVL}} and has become industry standard. We use it in all arithmetic operations to catch over/under flows.

\subsubsection*{2.2. Re-entrancy}
It is among high severity vulnerabilities that resulted the attack on DAO\footnote{It was a form of investor-directed venture capital fund to facilitate fundraising on new ideas or new projects through crowdfunding; providing the owners with tokens, which then enable them to vote for their favorite ideas and projects. \url{https://github.com/slockit/DAO.}} in 2016. An attacker could manage to drain US\$50M off the token funds \cite{DAO1}, \cite{DAO2}. ERC20 tokens would also be vulnerable to this attack if exchanging tokens for ETH is supported. An attacker can call the exchange function (e.g., \texttt{sell(tokens)}) to sell token and get back equivalent in ETH. However, before reaching to the end of the function and updating balances, the function might transfer control to the caller which allows the same function to be invoked over and over within the same transaction. This can be continued until draining all ETH of the token contract. The attack is known as same-function re-entrancy and could have three variants: Cross-function re-entrancy, Delegated re-entrancy and Create-Based re-entrancy\cite{SEREUM}. Mutex\cite{WiKiMutex} or CEI\cite{SolidtyDocSec} techniques can be used to prevent it. In Mutex, a state variable is used to lock/unlock transferred ETH by the lock owner (\ie token contract). The lock variable fails subsequent calls until finishing the first call and changing requester balance. CEI updates the requester balance before transferring any fund. All interactions (\ie external calls) happen at the end of the function and prevents recursive calls. Although CEI does not require a state variable and consumes less Gas, we use Mutex in addition to CEI. This protects token contract against Cross-function re-entrancy when attacker calls a different function than the initial function. In the proposal, \texttt{noReentrancy} modifier enforces Mutex and CEI is considered in the implementation of critical functions.

\subsubsection*{2.3. Unchecked return values}
In Solidity, sending ETH to external addresses are commonly performed by: (1) \texttt{call.value()}, (2) \texttt{transfer()}\footnote{\url{http://bit.ly/39C3x01}} or (3) \texttt{send()}. The \texttt{transfer()} method reverts all changes if the external call fails \cite{SoliditySendEther}. Other two methods are simply return a boolean value and manual check is required to revert transaction to the initial state. Before \textit{Istanbul} hard fork\cite{IstanbulUpgrades}, \texttt{transfer()} was the preferred way of sending ETH by forwarding only 2300 Gas. It prevents recursive calls and mitigates re-entrancy attack. EIP-1884\footnote{\url{http://bit.ly/2U2sHi3}} has increased Gas cost of some opcodes that fails this method\footnote{After \textit{Istanbul} hard-fork, \texttt{fallback()} function consumes more than 2300 Gas if called via \texttt{transfer()} or \texttt{send()} methods.}. The best practice is now not to rely on Gas and use \texttt{call.value()} method\cite{ConsensysStopTran}, \cite{ChainSecurity}. Since all remaining Gas will be sent by this command, one of re-entrancy mitigations (\ie Mutex or CEI) must be considered. We use \texttt{call.value()} in \texttt{sell()} and \texttt{withdraw()} functions and check the returned value to revert failed fund transfers.

\subsubsection*{2.4. Balance manipulation}
General assumption to receive ETH by smart contracts is via payable functions\footnote{\url{http://bit.ly/38FRRrQ}} (\ie \texttt{receive()}, \texttt{fallback()}, \etc), however, it is possible to send ETH without triggering payable functions, for example via \texttt{selfdestruct(contractAddress)} that is initiated by another contract. This allows forcing ETH and manipulate contract balance\cite{UnexpectedEth}. Hence, using checks like \texttt{address(this).balance} provides a relative security risk. To prevent exploiting this vulnerability, contract logic should avoid using exact values of the contract balance and keeps track of the known deposited ETH by a new state variable. Although we use \texttt{address(this).balance} in our implementation, but we do not check exact value of it (\ie  \texttt{address(this).balance == 0.5 ether)}. We only check whether the contract has enough ETH to send out or not. Therefore, there is no need to use a new state variable and consume more Gas to track contract's ETH. However, for developers who are interested to track it manually, there would be \texttt{contractBalance} variable to use. Two complementary functions are also considered to get current contract balance and check unexpected received ETH (\ie \texttt{getContractBalance()} and \texttt{unexpectedEther()}).

\subsubsection*{2.5. Public visibility}
In Solidity, visibility of functions are \texttt{Public} by default and they can be called by any external user/contract. It is recommended to always specify the visibility of all functions to prevent attacks like what happened to Parity MultiSig Wallet \cite{ParityFirstHack}. An attacker was able to call public functions and reset the ownership address of the contract. It caused draining of the wallets to the tune of \$31M. To prevent such attacks, we explicitly define visibility of each function. Interactive functions (\eg \texttt{Approve()}, \texttt{Transfer()}, \etc) are publicly accessible per specifications of ERC20 standard.

\subsubsection*{2.6. Multiple withdrawal}
This protocol-level issue was originally raised in 2017 \cite{MikVlad}, \cite{TomHale} and originating from ERC20 definition. It can be considered as \textit{Transaction-ordering} or \cite{OrderingAttack} or \textit{Front-running} \cite{eskandari2019sok} attack. There are two functions (\ie \texttt{Approve()} and \texttt{transferFrom()}) that can be used to authorize a third party for transferring tokens on behalf of someone else. Using these functions  in an undesirable situation (\ie Front-running or race-condition\footnote{Performing two or more operations at the same time due to nature of the blockchain.}) could result in condition that allows attacker to transfer more tokens than the owner ever wanted. There are several suggestions to mitigate this attack, however, securing \texttt{transferFrom()} method is the effective one while adhering specifications of the ERC20 standard\cite{ERC20MWA}. We added a new state variable to the \texttt{transferFrom()} function to track transferred tokens and mitigate the attack. 

\subsubsection*{2.7. State variable manipulation}
\texttt{DELEGATECALL} opcode in Ethereum enables to invoke external functions and execute them in the context of calling contract (\ie Invoked function can modify state variables of the caller). This makes it possible to deploy libraries once and reuse the code in different contracts. However, ability to manipulate internal state variables by external functions can lead to hijacking of the entire contract as it happened in Parity Multisig Wallet \cite{ParitySecondHack}. Preventive technique is to use \texttt{Library} keyword in Solidity to force the code to be stateless\footnote{Data are passed as inputs to functions and passed back as outputs. Libraries do not have any storage that makes such attacks unlikely.} \cite{LIB1}. There are two types of Library: Embedded and Linked. Embedded libraries have only internal functions, in contrast to linked libraries that have public or external functions. Deployment of linked libraries generates a unique address on the blockchain while the code of embedded libraries will be added to the contract's code \cite{LIB2}. In the proposal, there is only one library, \texttt{SafeMath}, that is defined as embedded library. We use \texttt{Library} keyword to declare it and has only internal functions. Therefor its code will be added to the ERC20 contract's code and EVM uses \texttt{JUMP} statement instead of \texttt{DELEGATECALL}. As a result, we do not use \texttt{DELEGATECALL} and will be safe to this vulnerability.

\subsubsection*{2.8. Frozen Ether}
Smart contracts can receive ETH similar to user accounts. In order to send the received ETH out of the contract, it is necessary to use withdrawal functions, so that the ETH does not get stuck in the contract as it happened in the case of Parity Wallet \cite{ParityWalletHack}. We define \texttt{withdraw()} function which allows the owner to transfer ETH out of the contract. The \texttt{sale()} function also makes it possible to transfer ETH during token exchange.

\subsubsection*{2.9. Unprotected SELFDESTRUCT}
As it happened in Parity wallet \cite{ParitySecondHack}, Self-destruct method is used to kill the contract and associated storage. It is recommended to get approval by multiple parties before running the method. We do not use this method and the deployed ERC20 tokens will be active on the blockchain forever.

\subsubsection*{2.10. Unprotected Ether Withdrawal}
Improper access control may allow unauthorized persons to withdraw ETH from smart contracts (as it happened in Rubixi\footnote{\url{https://bit.ly/2yrYP7P}}). Therefore, withdrawals must be triggered by only authorized accounts. We use \texttt{onlyOwner} modifier to enforce authentication on \texttt{withdraw()} function before sending out any funds.