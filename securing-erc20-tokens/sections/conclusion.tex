% !TEX root = ../main.tex

\section{Conclusion}

The development of smart contracts has proven to be error-prone in practice, and as a result, contracts deployed on public platforms are often riddled with security vulnerabilities. Exploited by the attackers, these vulnerabilities can often lead to major security incidents which introduce great cost due to the immutability characteristics of the blockchain technology. In this paper, we examine \erc security vulnerabilities and thoroughly discuss the technical details, the circumstances of the incidents together with their impacts and mitigations. We also integrate best practices to improve efficiency and productivity of the token. Eventually, we propose a secure \erc code that is not vulnerable to any of the attacks. Using auditing tools and comparing with the top ten \erc tokens shows the security of the proposal. It can be used as template to deploy new \erc tokens, migrate current vulnerable deployments or develop tools to automate auditing of \erc tokens.



