% !TEX root = ../main.tex

\section{Proposal}\label{sec:proposal}
Considering discussed vulnerabilities in section~\ref{sec:vul}, we propose as secure ERC20 code that is not vulnerable to any of them. It has been deployed on the Mainnet and the Solidity code is available on Etherscan\footnote{\url{https://bit.ly/2xvpnoh}}. Developers can refer to each mitigation technique separately to address a specific attack in their customized version. Required comments have been also added to clarify usage of each part. Standard functionalities of the token (\ie \texttt{approve()}, \texttt{transfer()}, \etc) have been tested by MetaMask\footnote{An extension for accessing Ethereum enabled distributed applications in the browser. The extension injects the Ethereum web3 API into every website's JavaScript context, so that DApps can read from the blockchain.\url{https://metamask.io/}} and no issue were raised. It could interact with the token successfully\footnote{\url{http://bit.ly/2IZYzPf}} and triggers expected events\footnote{\url{http://bit.ly/2Ub3vG9}} after transferring and receiving tokens. In addition to standard ERC20 methods, we introduce the following complementary features:

\begin{enumerate}
	\item \textbf{Selling tokens:} By using \texttt{sell()} function, token holders can send back tokens to the contract and receive ETH in return. Received ETH is based on the current exchange rate which is managed by \texttt{exchangeRate} variable. By default, this rate is 100 tokens for 1 ETH. For example, if someone sends 200 tokens to the contract, the contract sends back 2 ETH. After each exchange, \texttt{Sell} event tracks exchanged tokens. This feature is a financial advantage for new ERC20 tokens and reduces buyers doubts. They can return purchased token at any time and receive the equivalent in ETH. Another option for them is to wait for the token to be listed by crypto-exchanges (if it ever happens). Otherwise, they would not be able to exchange tokens if this feature is not support by the token contract.
	\item \textbf{Buying tokens:} Users can call \texttt{buy()} function to purchase autonomously tokens. This function is defined as \textit{payable}\footnote{Payable functions provide a mechanism to collect/receive ETH.} and accepts ETH. It calculates the equivalent of tokens based on the current exchange rate. It then increases balance of the buyer and logs \texttt{Buy} event for tracking of purchased tokens.
	\item \textbf{Withdrawing Ether:} This function can be called only by the contract owner. Since the contract accepts ETH, token owner may use \texttt{withdraw()} function to transfer ETH out of the contract. Otherwise, received ETH get stuck in the contract and would not be transferable. Transferring ETH out of the contract logs \texttt{Withdrawal} event.
\end{enumerate}

\noindent Supporting these extra features would be a financial advantage for new tokens and makes them independent of crypto-exchanges. All the required functionalities are directly supported by the token contract and no additional external services are required.